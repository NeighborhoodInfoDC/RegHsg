\documentclass[]{article}
\usepackage{lmodern}
\usepackage{amssymb,amsmath}
\usepackage{ifxetex,ifluatex}
\usepackage{fixltx2e} % provides \textsubscript
\ifnum 0\ifxetex 1\fi\ifluatex 1\fi=0 % if pdftex
  \usepackage[T1]{fontenc}
  \usepackage[utf8]{inputenc}
\else % if luatex or xelatex
  \ifxetex
    \usepackage{mathspec}
  \else
    \usepackage{fontspec}
  \fi
  \defaultfontfeatures{Ligatures=TeX,Scale=MatchLowercase}
\fi
% use upquote if available, for straight quotes in verbatim environments
\IfFileExists{upquote.sty}{\usepackage{upquote}}{}
% use microtype if available
\IfFileExists{microtype.sty}{%
\usepackage{microtype}
\UseMicrotypeSet[protrusion]{basicmath} % disable protrusion for tt fonts
}{}
\usepackage[margin=1in]{geometry}
\usepackage{hyperref}
\hypersetup{unicode=true,
            pdftitle={Regional Housing Framework},
            pdfborder={0 0 0},
            breaklinks=true}
\urlstyle{same}  % don't use monospace font for urls
\usepackage{color}
\usepackage{fancyvrb}
\newcommand{\VerbBar}{|}
\newcommand{\VERB}{\Verb[commandchars=\\\{\}]}
\DefineVerbatimEnvironment{Highlighting}{Verbatim}{commandchars=\\\{\}}
% Add ',fontsize=\small' for more characters per line
\usepackage{framed}
\definecolor{shadecolor}{RGB}{248,248,248}
\newenvironment{Shaded}{\begin{snugshade}}{\end{snugshade}}
\newcommand{\KeywordTok}[1]{\textcolor[rgb]{0.13,0.29,0.53}{\textbf{#1}}}
\newcommand{\DataTypeTok}[1]{\textcolor[rgb]{0.13,0.29,0.53}{#1}}
\newcommand{\DecValTok}[1]{\textcolor[rgb]{0.00,0.00,0.81}{#1}}
\newcommand{\BaseNTok}[1]{\textcolor[rgb]{0.00,0.00,0.81}{#1}}
\newcommand{\FloatTok}[1]{\textcolor[rgb]{0.00,0.00,0.81}{#1}}
\newcommand{\ConstantTok}[1]{\textcolor[rgb]{0.00,0.00,0.00}{#1}}
\newcommand{\CharTok}[1]{\textcolor[rgb]{0.31,0.60,0.02}{#1}}
\newcommand{\SpecialCharTok}[1]{\textcolor[rgb]{0.00,0.00,0.00}{#1}}
\newcommand{\StringTok}[1]{\textcolor[rgb]{0.31,0.60,0.02}{#1}}
\newcommand{\VerbatimStringTok}[1]{\textcolor[rgb]{0.31,0.60,0.02}{#1}}
\newcommand{\SpecialStringTok}[1]{\textcolor[rgb]{0.31,0.60,0.02}{#1}}
\newcommand{\ImportTok}[1]{#1}
\newcommand{\CommentTok}[1]{\textcolor[rgb]{0.56,0.35,0.01}{\textit{#1}}}
\newcommand{\DocumentationTok}[1]{\textcolor[rgb]{0.56,0.35,0.01}{\textbf{\textit{#1}}}}
\newcommand{\AnnotationTok}[1]{\textcolor[rgb]{0.56,0.35,0.01}{\textbf{\textit{#1}}}}
\newcommand{\CommentVarTok}[1]{\textcolor[rgb]{0.56,0.35,0.01}{\textbf{\textit{#1}}}}
\newcommand{\OtherTok}[1]{\textcolor[rgb]{0.56,0.35,0.01}{#1}}
\newcommand{\FunctionTok}[1]{\textcolor[rgb]{0.00,0.00,0.00}{#1}}
\newcommand{\VariableTok}[1]{\textcolor[rgb]{0.00,0.00,0.00}{#1}}
\newcommand{\ControlFlowTok}[1]{\textcolor[rgb]{0.13,0.29,0.53}{\textbf{#1}}}
\newcommand{\OperatorTok}[1]{\textcolor[rgb]{0.81,0.36,0.00}{\textbf{#1}}}
\newcommand{\BuiltInTok}[1]{#1}
\newcommand{\ExtensionTok}[1]{#1}
\newcommand{\PreprocessorTok}[1]{\textcolor[rgb]{0.56,0.35,0.01}{\textit{#1}}}
\newcommand{\AttributeTok}[1]{\textcolor[rgb]{0.77,0.63,0.00}{#1}}
\newcommand{\RegionMarkerTok}[1]{#1}
\newcommand{\InformationTok}[1]{\textcolor[rgb]{0.56,0.35,0.01}{\textbf{\textit{#1}}}}
\newcommand{\WarningTok}[1]{\textcolor[rgb]{0.56,0.35,0.01}{\textbf{\textit{#1}}}}
\newcommand{\AlertTok}[1]{\textcolor[rgb]{0.94,0.16,0.16}{#1}}
\newcommand{\ErrorTok}[1]{\textcolor[rgb]{0.64,0.00,0.00}{\textbf{#1}}}
\newcommand{\NormalTok}[1]{#1}
\usepackage{graphicx,grffile}
\makeatletter
\def\maxwidth{\ifdim\Gin@nat@width>\linewidth\linewidth\else\Gin@nat@width\fi}
\def\maxheight{\ifdim\Gin@nat@height>\textheight\textheight\else\Gin@nat@height\fi}
\makeatother
% Scale images if necessary, so that they will not overflow the page
% margins by default, and it is still possible to overwrite the defaults
% using explicit options in \includegraphics[width, height, ...]{}
\setkeys{Gin}{width=\maxwidth,height=\maxheight,keepaspectratio}
\IfFileExists{parskip.sty}{%
\usepackage{parskip}
}{% else
\setlength{\parindent}{0pt}
\setlength{\parskip}{6pt plus 2pt minus 1pt}
}
\setlength{\emergencystretch}{3em}  % prevent overfull lines
\providecommand{\tightlist}{%
  \setlength{\itemsep}{0pt}\setlength{\parskip}{0pt}}
\setcounter{secnumdepth}{0}
% Redefines (sub)paragraphs to behave more like sections
\ifx\paragraph\undefined\else
\let\oldparagraph\paragraph
\renewcommand{\paragraph}[1]{\oldparagraph{#1}\mbox{}}
\fi
\ifx\subparagraph\undefined\else
\let\oldsubparagraph\subparagraph
\renewcommand{\subparagraph}[1]{\oldsubparagraph{#1}\mbox{}}
\fi

%%% Use protect on footnotes to avoid problems with footnotes in titles
\let\rmarkdownfootnote\footnote%
\def\footnote{\protect\rmarkdownfootnote}

%%% Change title format to be more compact
\usepackage{titling}

% Create subtitle command for use in maketitle
\newcommand{\subtitle}[1]{
  \posttitle{
    \begin{center}\large#1\end{center}
    }
}

\setlength{\droptitle}{-2em}

  \title{Regional Housing Framework}
    \pretitle{\vspace{\droptitle}\centering\huge}
  \posttitle{\par}
  \subtitle{Pre-clean Loudoun County public records data}
  \author{}
    \preauthor{}\postauthor{}
    \date{}
    \predate{}\postdate{}
  

\begin{document}
\maketitle

{
\setcounter{tocdepth}{2}
\tableofcontents
}
Library: RegHsg

Project: Regional Housing Framework

Author: Sarah Strochak

Version: R 3.5.1, RStudio 1.1.423

Last updated February 04, 2019

Environment: Local Windows session (desktop)

\subsection{Description}\label{description}

The purpose of this program is to obtain additional data from the county
to supplement the Black Knight data when possible.

\subsection{Set-up}\label{set-up}

Load libraries and functions

\begin{Shaded}
\begin{Highlighting}[]
\KeywordTok{library}\NormalTok{(tidyverse)}

\KeywordTok{source}\NormalTok{(}\StringTok{"../Macros/read-bk.R"}\NormalTok{)}
\KeywordTok{source}\NormalTok{(}\StringTok{"../Macros/filter-bk.R"}\NormalTok{)}
\KeywordTok{source}\NormalTok{(}\StringTok{"../Macros/select-vars.R"}\NormalTok{)}
\end{Highlighting}
\end{Shaded}

Set FIPS code, filepath name, and directory for data storage (on L
drive)

\begin{Shaded}
\begin{Highlighting}[]
\NormalTok{currentfips <-}\StringTok{ "51107"}
\NormalTok{filepath <-}\StringTok{ "loudoun"}
\NormalTok{jdir <-}\StringTok{ }\KeywordTok{paste0}\NormalTok{(}\StringTok{"L:/Libraries/RegHsg/Data/"}\NormalTok{, filepath, }\StringTok{"/"}\NormalTok{)}
\NormalTok{rdir <-}\StringTok{ }\KeywordTok{paste0}\NormalTok{(}\StringTok{"L:/Libraries/RegHsg/Raw/"}\NormalTok{, filepath, }\StringTok{"/"}\NormalTok{)}
\end{Highlighting}
\end{Shaded}

Create directory for data exports

\begin{Shaded}
\begin{Highlighting}[]
\ControlFlowTok{if}\NormalTok{ (}\OperatorTok{!}\KeywordTok{dir.exists}\NormalTok{(}\StringTok{"../Data"}\NormalTok{)) \{}
  \KeywordTok{dir.create}\NormalTok{(}\StringTok{"../Data"}\NormalTok{)}
\NormalTok{\}}

\ControlFlowTok{if}\NormalTok{ (}\OperatorTok{!}\KeywordTok{dir.exists}\NormalTok{(}\KeywordTok{paste0}\NormalTok{(}\StringTok{"L:/Libraries/RegHsg/Raw/"}\NormalTok{, filepath))) \{}
  \KeywordTok{dir.create}\NormalTok{(}\KeywordTok{paste0}\NormalTok{(}\StringTok{"L:/Libraries/RegHsg/Raw/"}\NormalTok{, filepath))}
\NormalTok{\}}
\end{Highlighting}
\end{Shaded}

Load in Black Knight data for the region, select jurisdiction and
standard variables

\begin{Shaded}
\begin{Highlighting}[]
\ControlFlowTok{if}\NormalTok{ (}\OperatorTok{!}\KeywordTok{exists}\NormalTok{(}\StringTok{"region"}\NormalTok{)) \{}
\NormalTok{  region <-}\StringTok{ }\KeywordTok{read_bk}\NormalTok{(}\StringTok{"dc-cog-assessment_20181228.csv"}\NormalTok{)}
\NormalTok{\} }\ControlFlowTok{else}\NormalTok{ \{}
  \KeywordTok{warning}\NormalTok{(}\StringTok{"region data already read in"}\NormalTok{)}
\NormalTok{\}}

\NormalTok{jur <-}\StringTok{ }\NormalTok{region }\OperatorTok\StringTok{ }
\StringTok{  }\KeywordTok{filter_bk}\NormalTok{(}\DataTypeTok{fips =}\NormalTok{ currentfips) }\OperatorTok\StringTok{ }
\StringTok{  }\KeywordTok{select_vars}\NormalTok{()}
\end{Highlighting}
\end{Shaded}

\subsection{Download files}\label{download-files}

Loudoun has files on Loudoun's parcels and landuse existing parcels,
among others, but I selected these two.

\begin{Shaded}
\begin{Highlighting}[]
\NormalTok{pafile <-}\StringTok{ }\KeywordTok{paste0}\NormalTok{(rdir, filepath, }\StringTok{"-parcel-file.csv"}\NormalTok{)}
\NormalTok{prfile <-}\StringTok{ }\KeywordTok{paste0}\NormalTok{(rdir, filepath, }\StringTok{"-property-file.csv"}\NormalTok{)}

\CommentTok{# parcel file}
\ControlFlowTok{if}\NormalTok{ (}\OperatorTok{!}\KeywordTok{file.exists}\NormalTok{(pafile)) \{}
  \KeywordTok{download.file}\NormalTok{(}\StringTok{"file:///L:/Libraries/RegHsg/Raw/loudoun/loudoun-parcel-file.csv"}\NormalTok{,}
                \DataTypeTok{destfile =}\NormalTok{ pafile)}
\NormalTok{\}}

\CommentTok{# property file}
\ControlFlowTok{if}\NormalTok{ (}\OperatorTok{!}\KeywordTok{file.exists}\NormalTok{(prfile)) \{}
  \KeywordTok{download.file}\NormalTok{(}\StringTok{"file:///L:/Libraries/RegHsg/Raw/loudoun/loudoun-property-file.csv"}\NormalTok{,}
                \DataTypeTok{destfile =}\NormalTok{ prfile)}
\NormalTok{\}}
\end{Highlighting}
\end{Shaded}

\subsection{check to see if variables prfile and pafile
exist}\label{check-to-see-if-variables-prfile-and-pafile-exist}

\begin{Shaded}
\begin{Highlighting}[]
\KeywordTok{ls}\NormalTok{()}
\end{Highlighting}
\end{Shaded}

\begin{verbatim}
##  [1] "currentfips" "filepath"    "filter_bk"   "jdir"        "jur"        
##  [6] "pafile"      "prfile"      "rdir"        "read_bk"     "region"     
## [11] "select_vars"
\end{verbatim}

\subsection{Read files}\label{read-files}

\begin{Shaded}
\begin{Highlighting}[]
\NormalTok{property <-}\StringTok{ }\KeywordTok{read_csv}\NormalTok{(prfile,}
                     \DataTypeTok{col_types =} \KeywordTok{cols}\NormalTok{(}
  \DataTypeTok{OBJECTID =} \KeywordTok{col_double}\NormalTok{(),}
  \DataTypeTok{LU_PIN =} \KeywordTok{col_double}\NormalTok{(),}
  \DataTypeTok{LU_USE =} \KeywordTok{col_character}\NormalTok{(),}
  \DataTypeTok{LU_MULTI_USE_SUBCD1 =} \KeywordTok{col_character}\NormalTok{(),}
  \DataTypeTok{LU_AGE_RESTRICT =} \KeywordTok{col_character}\NormalTok{(),}
  \DataTypeTok{LU_HUNITS_EXIST =} \KeywordTok{col_double}\NormalTok{(),}
  \DataTypeTok{LU_GIS_ACRE =} \KeywordTok{col_double}\NormalTok{(),}
  \DataTypeTok{LU_LEGAL_ACRE =} \KeywordTok{col_double}\NormalTok{(),}
  \DataTypeTok{LU_DEVELOPABLE_PCT =} \KeywordTok{col_double}\NormalTok{(),}
  \DataTypeTok{LU_NON_RES_SQ_FT =} \KeywordTok{col_double}\NormalTok{(),}
  \DataTypeTok{LU_AS_OF_DATE =} \KeywordTok{col_character}\NormalTok{(),}
  \DataTypeTok{LU_DISPLAY =} \KeywordTok{col_character}\NormalTok{(),}
  \DataTypeTok{LU_TYPE =} \KeywordTok{col_character}\NormalTok{(),}
  \DataTypeTok{LU_USE_CONSOLIDATED =} \KeywordTok{col_character}\NormalTok{()}
\NormalTok{)}
\NormalTok{)}

\KeywordTok{rm}\NormalTok{(prfile)}
\end{Highlighting}
\end{Shaded}

\subsection{Clean and merge}\label{clean-and-merge}

Select variables from each dataset that we want to keep.

From the parcel file, we will keep the area of the parcel shapefile to
use in the event that the Black Knight lot size is unavailable or
incorrect. We also archive the raw parcel ID and create a new version
that will merge with the Black Knight data.

While Sarah says to complete this step, I skip it for now. I can always
go back and do this later, but I don't think it will serve me well to
get rid of variables until I know what I need.


\end{document}
